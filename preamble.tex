\usepackage{mwe}
\usepackage{graphicx}
\graphicspath{{figures/}}
\usepackage{color}
\definecolor{highlight}{rgb}{1,1,0.6}
\definecolor{link}{rgb}{0.5,0.0,0.0}
\definecolor{cite}{rgb}{0.0,0.0,0.6}
\definecolor{url} {rgb}{0.3,0.0,0.3}
\definecolor{grey}{rgb}{0.3,0.3,0.3}

\usepackage[hidelinks]{hyperref}
\hypersetup{
	colorlinks,
	linkcolor={cite},
	citecolor={cite},
	urlcolor ={cite}
}

\usepackage{tabularx}
\usepackage{array}
\usepackage{booktabs} % for nice rules/lines in tables
\usepackage{arydshln} % for dashed lines in tables

\usepackage{soul} % for highlighting text
\usepackage{xspace}
\usepackage[shortcuts]{extdash}
\usepackage{relsize} % used in the \anote and \comment macros.

%% annotation commands %% 
\newcommand{\anote}[1]{{\leavevmode\smaller\itshape\color{red}\{#1\}}}
\sethlcolor{highlight}
\newcommand{\comment}[2]{\hl{#1} {{\leavevmode\smaller\color{red}\itshape\{#2\}}}}

%% PM Define authornote command for comments
\newcommand{\authornote}[1] {
	\begin{center}
		\framebox{
			{\begin{minipage}[t]{0.9\linewidth}
					\raggedright  \textbf{[PM]}~ \scriptsize #1 \normalsize
			\end{minipage}}
		}
	\end{center}
}

\usepackage{amssymb}
\usepackage{bm}
\usepackage{mathtools}




