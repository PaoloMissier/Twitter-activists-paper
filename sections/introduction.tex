\section{Introduction}

We present a customisable software framework for incrementally discovering and ranking individual profiles for classes of online users, through analysis of their social activity on Twitter.
Practical motivation for this work comes from our ongoing effort to support health officers in tropical countries, specifically in Brazil, in their fight against airborne virus epidemics like Dengue and Zika. Help from community activists is badly needed to supplement the scarce public resources deployed on the ground. Our past work has focused on identifying relevant content on Twitter that may point health authorities directly to mosquito breeding sites~\cite{Sousa2018}, as well as to users who have shown interest in those topics, i.e., by posting relevant content on Twitter~\cite{Missier2017}. 

The approach described in this paper generalises those past efforts, by attempting to discover users who demonstrate an inclination \textit{to become engaged in social issues, regardless of the specific topic}.
We refer to this class of users as \textit{activists}.
The rationale for this approach is that activists who manifest themselves online on a range of social initiatives, may be more sensitive to requests for help on specific issues than the average Twitter user.
In the paper we experiment with healthcare-related online campaigns in the UK. Application of the approach to our initial motivating case study in ongoing as part of a long-term collaboration, and is not specifically discussed in the paper.

To be clear, this work is not about providing a robust definition of online activism, or to demonstrate that online activism translates into actual engagement in the ``real world''.
%
Instead, we start by acknowledging that the notion of activist is not as well formalised in the literature as that of, for example, \textit{influencers}, and we develop a generic content processing pipeline which can be customised to identify a variety of classes of users. 
The pipeline repeatedly searches for and ranks Twitter user profiles by collecting quantitative network- and content-based user metrics. 
Once targeted to a specific topic, it provides a tool for exploring operational definitions of user roles, including online activism, i.e., by combining the metrics into higher level, \textit{engineered} user features to be used for ranking.

Although the user harvesting pipeline is generally applicable to the analysis of a variety of user profiles, our focus is on the search for a satisfactory operational definition of online activism. 
%
According to the Cambridge Dictionary, an \textit{activist} is ``A person who believes strongly in political or social change and takes part in activities such as public protests to try to make this happen''.
%
While activism is well-documented, e.g. in the social movement literature~\cite{doi:10.1080/14742830701497277}, and online activism is a well-known phenomenon \cite{IJoC1246}, research has been limited to the study of its broad societal impact. 
In contrast, we are interested in the fine-grained discovery of activists at the level of the single individual, that is, we seek people who feel passionate about a cause or topic, and who take action for it. 
Searching for online activists is a realistic goal, as activists presence in social media is widely acknowledged, and it is also clear that social media facilitates activists communication and organization \cite{Poell2014,Youmans2012}. 
Specific traits that characterise activists include awareness of causes and social topic and the organization of social gatherings and activities, including in emergency situations, by helping organize support efforts and diffusion of useful information.
 
 \anote{say something about bots either here or at the end?  
 	
 	refviewer says:
 - is there a bot analysis step? While institutional accounts might be ok, I think any meanginful analysis of twitter users and data should include a step that identifies and exclude robots and automated accounts. It might be that your pipeline is able to exclude them aoutomatically, but that would seem to me as a fortunate side- effect and not as a conscious choice.
}

\subsection{Challenges}
 
The definition of online activism translates into technical challenges in systematically harvesting suitable candidate users.
%
Firstly, the potentially more subdue nature of activists, relative to influencers, makes it particularly difficult to separate their online footprint from the background noise of general conversations.
Also, interesting activists are by their nature associated to specific topics and manifest their nature in local contexts, for instance as organisers or participants to local events. 
Finally, we expect their personal engagement to be sustained over time and across multiple such contexts. 
These observations suggest that the models and algorithms developed for influencers are not immediately applicable, because they mostly operate on global networks, where less prominent users have less of a chance to emerge.
Some topic-sensitive metrics and models have been proposed to measure social influence, for example, \textit{alpha centrality}~\cite{Bonacich2001,Overbey2013} and the \textit{Information Diffusion} model~\cite{Pal2011}. Algorithms based on topic models have also been proposed to account for topic specificity~\cite{Zhao2011b}. However, these approaches are still aimed at measuring influence, not activism, and assume a one-shot discovery process, as opposed to a continuous, incremental approach.

\subsection{Approach and Contributions}

To address these challenges, the approach we propose involves a two-fold strategy. 
Firstly, we identify suitable contexts that are topic-specific and limited both in time and, optionally, also in space, i.e., regional initiatives, events, or campaigns.
We then search for users only within these contexts, following the intuition that low-key users who produce weak online signal have a better chance to be discovered when the search is localised and then repeated across multiple such contexts.
By continuously discovering new contexts, we hope to incrementally build up a users' dataset where users who appear in multiple contexts are progressively more strongly characterised.
%
Secondly, to allow experimenting with varying technical definitions of \textit{activist}, we collect a number of network-based and content-based user profile features, mostly known from the literature, and make them available to experiment with a variety of user rankings.

The paper makes the following specific contributions.
%
Firstly, we propose a data processing pipeline for harvesting Twitter content and user profiles, based on multiple limited contexts. 
The pipeline includes community detection and network analysis algorithms aimed at discovering users within such limited contexts.

Secondly, we have implemented a comprehensive set of content-based metrics that results into an ever-growing database of user profile features, which can then be used for mining purposes. 
User profiles are updated when they are repeatedly found in multiple contexts.

Lastly, for empirical evaluation of our implementation, we demonstrate an operational definition of the activist profile, defined in terms of the features available in the database. We collected about 3,500 users  across 25 contexts in the domain of healthcare awareness campaigns in the UK during 2018, and demonstrated three separate ranking functions, showing that it is possible to identify individuals as opposed to well-known organisations.
The application of the approach to the specific challenge of combating tropical disease epidemics in Brazil is currently in progress and is not reported in this paper.

\subsection{Related Work}  \label{sec:related}


\anote{add the following: }

\cite{7569535} Proposes a supervised regression algorithm to rank influence of Twitter users. It uses sensible features as they are not based on content, but the method performs poorly. This is because it would need a huge labelled dataset to work effectively.

\cite{6978961}Proposes a method to create Twitter user ontologies considering the content type of the tweets. This approach could be used to gain insights over a user, but fails to give a comprehensive description of the user activity as it is based only on recent user activity, also due to Twitter API limitations.

\anote{end new refs}


The closest body of research to this work is concerned with techniques for the discovery of online \textit{influencers}. 
% The characterisation of online activism given in the introduction is substantially different from that of influencers, which have received much attention in the literature.
According to~\cite{Kardara2015}, influencers are \textit{prominent individuals with special characteristics that enable them to	affect a disproportionately large number of their peers with their actions}.
%
A large number of metrics and techniques have been proposed to make this generic definition operational~\cite{RIQUELME2016949}. 
%
These metrics tend to favour high visibility users across global networks, regardless of their actual impact~\cite{Cha2010MeasuringUI}. 
%
In contrast, activists are typically low-key, less prominent users who only emerge from the crowd by signalling high levels of engagement with one or more specific topics, as opposed to being thought-leaders. 
%
While such behaviour can be described  using well-tested metrics~\cite{RIQUELME2016949}, it should also be clear that new ways to combine those metrics are required.

The algorithm proposed in \cite{MATIC2011} aims to identify influencers based on a single topic context, based on relevant social media conversations.
Metrics include number of ``likes'', viewers per months, post frequency and  number of comments per post, as well as the ratio of positive to negative posts.
As some of these metrics are qualitative and difficult to acquire, however, this approach is not easy to automate.
%
Another approach to ranking topic-specific influencers within specific events appears in  \cite{Kardara2015}, where network dynamics are accounted for in real-time.
Once again however, the effect is to discover users who receive much attention, but do not necessarily create a real impact over users inside one topic.

Machine learning is used in~\cite{Biran2012} to analyse posted content and recognise when a user is able to influence another  inside a conversation.
This however requires substantial a priori ground truth, making this approach impracticable in our case. In addition, the need to create a classifier for each topic limits the scalability of the system.

% Some approaches try to extract the content of conversations over topics to understand the ability of a user to be influential within a topic. , a classifier is trained to identify when an user is able to influence another one inside a conversation. However, in this case the necessity of creating labelled data to build a machine learning classifier makes this approach impracticable in our case, which aims to be most practical possible. 
	
% \anote{\cite{Kardara2015} KARDARA:
% 	Here a new ranking algorithm is proposed to find local influencers on Twitter that appear within the context of a specific event being discussed, incorporating the network dynamics as the event evolves with time. 
% 	Local to an event context but focus on user reputation. 
% 	The above techniques all attempt to define influence as
% 	some measurable attribute or observation in the network, such as how many times an original story appears on a website, or how centrally connected a node is within various defined modules within the networks.
% 	The definition used in this work for influence is “who is being listened to the most”.
% 	- this is not who we are looking for.
% 	- one single large event
% 	- no breakdown into communities
% }\\
Unlike the majority of the influencer ranking algorithms, in \cite{Schenk2011} a topic-specific influencer ranking is proposed. First it harvests sequentially timed snapshots of the network of users related to a topic. Then it ranks the users based on the number of followers gained and lost in the considered snapshots.

Finally, \cite{Bizid:2015:PUD:2808797.2809411} presents a model for identifying ``prominent users'' regarding a specific topic event in Twitter. Those are users who focus their attention and communication on the aforementioned topic event. Users are described by a feature vector, computed  in real-time, which allows a separation between on-topic and off-topic users activity over Twitter. Similar to~\cite{Biran2012}, problems of scalability and adaptability arise as two supervised learning methods are used, one to discriminate prominent users from the rest and the other to rank them.