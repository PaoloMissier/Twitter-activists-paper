\vspace{-10pt}
\section{Conclusions and lessons learnt}
\vspace{-10pt}


Motivated by the need to find an operational definition of ``online activists'' that is grounded in well-established network and user-activity metrics, we have designed a Twitter content processing pipeline for progressively harvesting Twitter users based on their engagement with online socially-minded events, or campaigns, which we have called \textit{contexts}.
The pipeline yields a growing database of user profiles along with their associated metrics, which can then be analysed to experiment with user-defined user ranking criteria. The pipeline is  designed to select promising candidate profiles, but the approach is unsupervised, i.e., no manual classification of example users is provided.
We have empirically evaluated the pipeline on a case study, along with experimental scoring functions to show the viability of the approach. 

The design of the pipeline show that useful harvesting of interesting users can be accomplished within the limitations imposed by Twitter on its APIs.
The next challenge is to automate the discovery of new contexts so that the pipeline may continuously add new users to the database, and update its metrics for an increasing number of  repeat users.
Only at this point will it be possible to   validate the entire approach, hopefully with help from third party users, on a variety of new context topics.