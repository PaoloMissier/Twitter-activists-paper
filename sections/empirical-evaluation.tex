\section{Empirical Evaluation} \label{sec:evaluation}

The typical approach to identifying specific classes on online users relies on expert-generated ground truth, i.e., to determine which users belong to the desired class. 
Such approach, however, is vulnerable to  the subjectivity of the experts, whereby the evalution would  be measuring the fit of the model to the specific experts' own opinion. 
In contrast, we follow an \textit{unsupervised} approach where there is no a priori knowledge of user relevance.  
We aim to demonstrate the value of our pipeline in creating a database of online profiles, pre-selected according to specific topological properties in order to filter out background noise,  along with a community-accepted set of engineered features that are ready to be mined using any user-defined function.

Thus, our evaluation (i) shows the pipeline in action on a significant set of contexts, and (ii) demonstrates useful ranking functions that operate on the resulting database.
Specifically, we  start by manually selecting homogeneous contexts from a single social domain, namely public healthcare campaigns, to demonstrate network construction, breakdown of each  context network into communities, and harvesting of users from each community, along with their metrics.
We then provide examples of ranking functions aimed at capturing the notion of  \textit{online activists}, and we empirically validate them by inspecting the profiles of representative top-k users.

\anote{implementation:

	all in python. using Pandas, NetworkX, Selenium public libraries. 
	
	deployed on Azure cloud, one node with standard commodity configuration.
}

\subsection{Contexts and networks}  \label{sec:contexts}
 
We have manually selected 25 contexts within the scope of health awareness campaigns in the UK, all occurring in 2018, all well-characterised using predefined hashtags.
Due to limitations imposed by Twitter on the number of posts that can be retrieved within a time interval, only $200$ tweets were retrieved from each context.
 Table~\ref{tab:contexts} lists the events along with key metrics for their corresponding user-user networks. 
To recall, \textit{assortativity} measures how frequently nodes are likely to connect to other nodes with the same degree ($>0$) or with a different degree ($<0$). 
Negative figures (mean: -0.22, std dev: 0.17) are in line with what is observed on the broader Twitter network~\cite{Fisher2017}.
%
The very small figures for density (mean: 0.004, std dev: 0.002), defined as $\frac{\#edges }{\mathit{\mathit{\#nodes}} (\mathit{\#nodes} -1)}$, suggest very few connections exist amongst users within a context. 
This makes it difficult to detect meaningful communities, as described below, thus for some context the topological metrics are measured on the entire network as opposed to within each community.
This view is also supported by the average node degree (mean: 2.04, std dev: 0.46) and the ratio of strongly connected components to the number of nodes (mean: 0.98, std. dev. 0.02).

\begin{table}
	%\anote{content for this comes from cell [5] and cell [6] table 2 and cell [8]}
	\resizebox{\textwidth}{!}{
	    \begin{tabularx}{\textwidth}{|X|P{2.2cm}|P{1.2cm}|P{1cm}|P{1.4cm}|P{1.2cm}|P{1.3cm}|}

\hline
\textbf{Context name} & \textbf{Period (2018)} & \textbf{Nodes} & \textbf{Edges} & \textbf{Density} & \textbf{Avg degree} & \textbf{Assor-tativity} \\ \hline
16 days of action & 11-25 / 12-10 & 396 & 349 & 0.002 & 1.8 & -0.1 \\ \hline
Elf day & 12-03 / 12-12 & 365 & 436 & 0.003 & 2.4 & -0.2 \\ \hline
Dry january & 01-01 / 01-31 & 235 & 234 & 0.004 & 2.0 & -0.3 \\ \hline
Cervical cancer prevention week & 01-21 / 01-27 & 209 & 192 & 0.004 & 1.8 & -0.1 \\ \hline
Time to talk day & 02-06 / 02-07 & 268 & 231 & 0.003 & 1.7 & -0.2 \\ \hline
Eating disorder awareness week & 02-25 / 03-03 & 256 & 241 & 0.004 & 1.9 & -0.2 \\ \hline
Rare disease day & 02-28 / 03-01 & 294 & 206 & 0.002 & 1.4 & -0.2 \\ \hline
Ovarian cancer awareness month & 03-01 / 03-31 & 215 & 202 & 0.004 & 1.9 & -0.4 \\ \hline
Nutrition and hydration week & 03-11 / 03-17 & 273 & 326 & 0.004 & 2.4 & -0.3 \\ \hline
Brain awareness week & 03-11 / 03-17 & 307 & 281 & 0.003 & 1.8 & -0.1 \\ \hline
No smoking day & 03-13 / 03-14 & 254 & 219 & 0.003 & 1.7 & -0.3 \\ \hline
Epilepsy awareness purple day & 03-26 / 03-27 & 306 & 252 & 0.003 & 1.6 & -0.2 \\ \hline
Experience of care week & 04-23 / 04-27 & 176 & 196 & 0.006 & 2.2 & -0.1 \\ \hline
Brain injury week & 05-01 / 05-31 & 238 & 306 & 0.005 & 2.6 & -0.1 \\ \hline
Mental health awareness week & 05-14 / 05-20 & 268 & 245 & 0.003 & 1.8 & -0.5 \\ \hline
Dementia action week & 05-21 / 05-31 & 300 & 300 & 0.003 & 2.0 & -0.0 \\ \hline
Mnd awareness month & 06-01 / 06-30 & 141 & 234 & 0.012 & 3.3 & -0.3 \\ \hline
Wear purple for jia & 06-01 / 06-30 & 165 & 245 & 0.009 & 3.0 & -0.5 \\ \hline
Carers week & 06-11 / 06-17 & 270 & 277 & 0.004 & 2.1 & 0.0 \\ \hline
National dementia carers & 09-09 / 09-10 & 184 & 177 & 0.005 & 1.9 & -0.2 \\ \hline
Mens health week & 06-11 / 06-17 & 264 & 214 & 0.003 & 1.6 & -0.2 \\ \hline
Stress awareness day & 11-07 / 11-08 & 293 & 209 & 0.002 & 1.4 & -0.2 \\ \hline
National dyslexia week & 10-01 / 10-07 & 229 & 235 & 0.004 & 2.1 & -0.2 \\ \hline
Ocd awareness week & 10-07 / 10-13 & 202 & 193 & 0.005 & 1.9 & -0.6 \\ \hline
Jeans for genes day & 09-21 / 09-22 & 246 & 325 & 0.005 & 2.6 & -0.2 \\ \hline

\end{tabularx}

	}
	%\anote{add the rest of the  data from the original tables}
	\caption{List of contexts used in the experiments along with network metrics \hl{and count} of \demon communities}
	\label{tab:contexts}
\end{table}

\subsection{Communities}  \label{sec:communities}

Next, we report on the effect of the two community detection algorithms mentioned earlier, namely \demon~and \infomap, on each of the networks. 
%
Regarding \demon, we find that, within each network, the algorithm  identifies \textit{meaningful} communities consisting of more than one single node in only 48\% of the networks, while for the remaining 52\% 
it does not detect any communities.
In this situation our pipeline creates only one community consisting of the entire network, which is used to calculate the users' in-degrees, and all users in the network are added to the database.
%
When meaningful communities are found, on the other hand, only users who belong to one of those community are added. 
These are about 6\% of the users on average, with an average of only 1.92 communities per network.
The penultimate column in Table~\ref{tab:contexts} reports the ratio of the number of communities to number of nodes in each of the networks. 
A value of zero indicates that no meaningful communities were used.
%
This approach results in 3570 users being added to the database.
The average assortativity of individual \demon~communities is slightly negative -0.28, in line with the average for their parent networks.

In contrast, the \infomap~ algorithm provides meaningful communities for all networks.
Those with fewer than 3 users are discarded, leaving  18.88 communities per network on average (as opposed to 1.92 for \demon).
The rightmost column in Table~\ref{tab:contexts} reports the number of communities normalised by the size of the network, where a community contains 8.5 users on average. 
Overall, this approach resulted in 3567 users being added to the database (on average 253 users per network).
The average assortativity across all communities is again slightly negative (-0.43).

\anote{conclude that \infomap~is preferrable and this is where the users in the database come from}

\anote{cell [10]: think about how to comment on histogram or omit altogether}

\subsection{Users discovery}  \label{sec:users}

Finally, from each context we have extracted users who are either part of a community, or in the case of non-meaningful communities, all users in the network.
Using these criteria, a total of 3567 users has been added to the database from 25 contexts.
We are interested in tracking the users' presence across more than one context, and in the specification of ranking functions that let specific user profiles emerge.

Regarding user presence, we found that there are 160 users who appear at least in two contexts. 
After community detection, only a fraction of these users survive. 
These are either users who belong to a meaningful community, or users who belong to a context network where communities could not be identified.

This filtering process results in 61 of the 160 users still seen as repeat users, while the remaining 99 are either removed altogether, or they only appear once. 
Out of these, 55 appear twice, and 2 appear three times, and 2 appear four times. 
Thus, only 1.6\% of users appear more than once when communities with more than 3 users are considered, compared to the overall 4.5\% repeat users.
%
Table~\ref{tab:repeat-users} reports the top repeat users along with their \textit{Follower Rank}.  \anote{comment on whether these are individuals or well-known organisations}
\anote{can we sort these by no-participation and follower-rank??}
%
Fig.~\ref{fig:repeat-users-frequency} shows the number of repeat users per context. 

\begin{table}
	\centering
	\framebox{\anote{content from cell [13]} }
	\resizebox{\textwidth}{!}{
		        \begin{tabularx}{\textwidth}{|X|X|X|X|X|Y|Y|}
\hline
\textbf{username} & \textbf{name} & \textbf{url} & \textbf{location} & \textbf{bio} & \textbf{follower rank} & \textbf{no participations} \\ \hline
alzheimerssoc & Alzheimer's Society & http://www.alzheimers.org.uk/ & England, Wales \& N.Ireland & We provide information and support, fund research and create lasting change for people affected by dementia. & 0.99 & 4 \\ \hline
dementiauk & Dementia UK & http://www.dementiauk.org & Aldgate, London & Dementia UK provides specialist dementia support for families through our Admiral Nurse service. We respond to messages 9-5 Monday to Friday on Twitter. & 0.98 & 4 \\ \hline
mentalhealth & Mental Health Fdn & http://www.mentalhealth.org.uk & UK & The UK's charity for everyone's mental health, promoting good mental health for all. Keep up to date with our work in Scotland by following & 0.97 & 3 \\ \hline
colesmillerllp & Coles Miller LLP & http://www.coles-miller.co.uk & Dorset & Dorset Solicitors in & 0.65 & 3 \\ \hline
rdash\_nhs & RDaSH NHS FT & http://www.rdash.nhs.uk & Doncaster & Rotherham Doncaster and South Humber NHS Foundation Trust (Monitored Mon - Fri 9am - 5pm, except Bank Holidays) & 0.88 & 2 \\ \hline
alzsocseengland & Alzheimer's Society - South East England & https://www.alzheimers.org.uk/homepage/169/our\_local\_offices &  & Alzheimer's Society ( & 0.64 & 2 \\ \hline
jeremy\_hunt & Jeremy Hunt & http://www.jeremyhunt.org &  & Foreign Secretary \& South West Surrey MP & 1.0 & 2 \\ \hline
nhsengland & NHS England & http://www.england.nhs.uk &  & Health and high quality care for all, now and for future generations. Please see our response policy: & 0.99 & 2 \\ \hline
carersuk & Carers UK & http://www.carersuk.org & United Kingdom & Caring for a relative or friend? When caring affects you and your family Carers UK is here to provide the support and advice you need. & 0.95 & 2 \\ \hline
mndassoc & MND Association & http://mndassociation.org & England, Wales \& Northern Ireland & Our vision is a world free from motor neurone disease & 0.64 & 2 \\ \hline
        \end{tabularx}
	}
	\caption{Top-k repeat users, amongst those identified as belonging to some community.}
	\label{tab:repeat-users}
\end{table}

\begin{figure*}
	\centering
	\includegraphics[width=1.2\linewidth]{figures/repeat-users-frequency}
	\caption{Number of repeat users for each context}
	\label{fig:repeat-users-frequency}
\end{figure*}

\subsection{Users ranking} \label{sec:ranking}

\anote{show the effect of one or two ranking functions and single out users that ``look like activists'' top empirically prove the point}