\section{Contexts and user metrics}

%Two sets of criteria are used to establish relevance.
%Firstly, a context defined by a combination of spatio-temporal and keyword / hashtag constraints to describe the social topics of interest, for instance ``social health care campaigns'' or ``Zika awareness day in Rio de Janeiro''.
%%
%Secondly, a set of metrics are specified to characterise the relevance of user profiles for a specific domain, along with a user-defined function that is specific to user roles, for instance ``activist'', to compose the features into a single value, i.e., a relevance score, which can then be used to rank user profiles both within and across contexts.
%The metrics are meant to capture some operational definition of relevance for specific kinds of user roles. 

The aim of the pipeline is to repeatedly and efficiently discover user profiles from the Twitter post history within user-specified contexts,\footnote{Our plan is automate context discovery in the next phase of this work.} and use the process to grow a database of feature-rich user profiles that can be used to rank users according to user-defined relevance functions. 
The criteria used to define contexts, profile relevance functions, and associated user relevance thresholds can be configured for specific applications.

\subsection{Contexts and Context networks} \label{sec:contexts}

As described In Sec.~\ref{sec:reference}, contexts are meant to identify events or campaigns around social issues, which are characterised by temporal boundaries and by hashtags and/or keyword terms, and optionally also by spatial constraints, i.e., specified using a geographical bounding box.
These are \textit{weak} contexts, because Twitter does not natively support the notion of event or campaign (unlike, for example, Facebook, Instagram, or Meetup).
We denote a generic context as
\begin{equation}
C = \langle s, [t_1, t_2], K \rangle 
\label{eq:context}
\end{equation}
where $s$ represents the optional spatial constraint, $[t_1, t_2]$ a time interval, and $K = \{ k_1 \dots k_n\}$ is the set of terms used to filter content within the spatio-temporal boundaries.
%
$C$ defines search criteria, which produce a set $P(C)$ of posts when submitted to Twitter.
We only consider two Twitter activities: an \textit{original tweet}, or a \textit{retweet}.
Let $u(p)$ be the user who originated a tweet $p \in P(C)$.
We say that both $p$ and $u(p)$ are \textit{within context} $C$.

We also define the complement $\Tilde{P}(C)$ of $P(C)$ as the set of posts found using the same spatio-temporal constraints, but which do not contain any of the terms in $K$. More precisely, given a context $C'= \langle s, [t_1, t_2], \emptyset \rangle$ with no terms constraints, we define $\Tilde{P}(C) = P(C') \setminus P(C)$. 
We refer to these posts, and their respective users, as ``out of context $C$''.

The set of posts $P(C)$ induces a user-user social network graph $G_C = (V,E)$ where $V$ is the set of all users who have authored any $p \in P(C)$: 
$V = \{ u(p) | p \in P(C) \}$, and a weighted edge $e = \langle u_1, u_2, w \rangle$ is added to $E$ for each pair of posts $p_1, p_2$ such that $u(p_1) = u_1, u(p_2) = u_2$ and 
either (i) $p_2$ is a retweet of $p_1$, or (ii) $p_1$ contains a mention of $u_2$.
For any such edge, the weight $w$ is a count of such pairs of posts occuring in $P(C)$ for the same pair of users.

\subsection{User relevance metrics}  \label{sec:metrics}

We support a number of metrics that are generally accepted by the community as forming a core, from which many different social user roles are derived~\cite{RIQUELME2016949}. 
We distinguish amongst three types of features, which differ in the way they are computed from the raw Twitter feed:
\begin{description}
	\item[Content-based metrics] that rely solely on content and not on the user-user graph topology. These metrics are defined relative to a topic of interest, i.e., a context;
	\item[Context-independent topological metrics] that encode context-independent, long-lived relationships amongst users, i.e., follower/followee; and 
	\item[Context-specific topological metrics] that encode user relationships that occur specifically within a context.
\end{description}

All metrics are functions of a few core features that can be directly extracted from Twitter posts. 
We start with a set of features that are commonly used in the literature as a baseline.
Given a context $C$ containing user $u$, we define:
%
\begin{align*}
\mathit{R1}(u) &\text{: Number of retweets by $u$, of tweets from other in-context users;}\\
\mathit{R2}(u)&\text{: Number of unique users in $C$, who have been retweeted by $u$;}\\
\mathit{R3}(u)&\text{: Number of retweets of $u$'s tweets;}\\
\mathit{R4}(u)&\text{: Number of unique users in $C$ who retweeted $u$'s tweets;}\\
\mathit{P1}(u)&\text{: Number of original posts by $u$ within $C$;}\\
\mathit{P2}(u)&\text{: Number of web links found in original posts by $u$ within $C$;} \\
\mathit{F1}(u)& \text{: Number of followers of $u$;}\\
\mathit{F2}(u)& \text{: Number of followees of $u$}
\end{align*}
%

Note that, given $C$, we can evaluate each of the features above with respect to either $P(C)$ or  $\Tilde{P}(C)$ independently from each other, that is, we can consider a ``on-context'' and an ``off-context'' version of each feature.
%
For example, we are going to write $R1_{on}(u)$ to denote the number of context retweets and $R1_{\mathit{off}}(u)$ the number of out-of-context retweets by $u$, i.e., these are retweets that occur within $C$'s spatio-temporal boundaries, but do not contain any of the hashtags or keywords that define $C$.  
%
We similarly qualify all other features.
%
Using these core features, we have implemented the following metrics.

For \textbf{content-based metrics}, we have:
\begin{align}
\textit{Topical Focus:~\cite{Missier2017}:} ~ \mathit{TF}(u) & =  \frac{\mathit{P1}_{\mathit{on}}(u)}{\mathit{P1}_{\mathit{off}}(u) +1}    \label{eq:TF}\\
\textit{Topical Strength~\cite{Bizid2018}:} ~\mathit{TS}(u) & =	\frac{\mathit{P2}_{\mathit{on}}(u) \cdot \log(\mathit{P2}_{\mathit{on}}(u) + R3_{\mathit{on}} +1 )}{\mathit{P2}_{\mathit{off}}(u) \cdot \log(\mathit{P2}_{\mathit{off}}(u) + R3_{\mathit{off}} +1 ) + 1}   \label{eq:TS} \\
\textit{Topical Attachment~\cite{Bizid:2015,Poell2014}:} ~\mathit{TA}(u) & = \frac{\mathit{P1}_{\mathit{on}}(u) + \mathit{P2}_{\mathit{on}}(u)}{\mathit{P1}_{\mathit{off}}(u) + \mathit{P2}_{\mathit{off}}(u) +1} \label{eq:TA}
\end{align}

We implement one \textbf{Context-independent topological metric} and one \textbf{Context-specific topological metric}, both commonly used, see e.g.~\cite{RIQUELME2016949}:
\begin{align}
\textit{Follower Rank:}  \quad \mathit{FR}(u) = \frac{\mathit{F1}(u)}{\mathit{F1}(u)+\mathit{F2}(u)}   \label{eq:FR}\\
\textit{In-degree centrality:} \quad \mathit{IC}(u) = \frac{\mathit{indegree}(u)}{N-1}  \label{eq:IDC}
\end{align}
where $N$ is the number of nodes in the network induced by $C$.

Note that the metrics we have selected are a superset of those indicated in recent studies on online activism, namely \cite{Lotan2011} and \cite{Poell2014}, and thus support our empirical evaluation, described in Sec.~\ref{sec:evaluation}.

\anote{Summary of user features if space allows}