% This is samplepaper.tex, a sample chapter demonstrating the
% LLNCS macro package for Springer Computer Science proceedings;
% Version 2.20 of 2017/10/04
%
\documentclass[runningheads]{llncs}
%
\usepackage{mwe}
\usepackage{graphicx}
\graphicspath{{figures/}}
\usepackage{color}
\definecolor{highlight}{rgb}{1,1,0.6}
\definecolor{link}{rgb}{0.5,0.0,0.0}
\definecolor{cite}{rgb}{0.0,0.0,0.6}
\definecolor{url} {rgb}{0.3,0.0,0.3}
\definecolor{grey}{rgb}{0.3,0.3,0.3}

\usepackage[hidelinks]{hyperref}
\hypersetup{
	colorlinks,
	linkcolor={cite},
	citecolor={cite},
	urlcolor ={cite}
}

\usepackage{tabularx}
\usepackage{graphicx}
\usepackage{array}
\usepackage{booktabs} % for nice rules/lines in tables
\usepackage{arydshln} % for dashed lines in tables

\usepackage{soul} % for highlighting text
\usepackage{xspace}
\usepackage[shortcuts]{extdash}
\usepackage{relsize} % used in the \anote and \comment macros.

%% annotation commands %% 
\newcommand{\anote}[1]{{\leavevmode\smaller\itshape\color{red}\{#1\}}}
\sethlcolor{highlight}
\newcommand{\comment}[2]{\hl{#1} {{\leavevmode\smaller\color{red}\itshape\{#2\}}}}

%% PM Define authornote command for comments
\newcommand{\authornote}[1] {
	\begin{center}
		\framebox{
			{\begin{minipage}[t]{0.9\linewidth}
					\raggedright  \textbf{[PM]}~ \scriptsize #1 \normalsize
			\end{minipage}}
		}
	\end{center}
}


\newcommand{\demon}{{DEMON}~}
\usepackage{amssymb}
\usepackage{bm}
\usepackage{mathtools}






\begin{document}
%
\title{Fantastic activists and where to find them\thanks{}}
%
%\titlerunning{Abbreviated paper title}
% If the paper title is too long for the running head, you can set
% an abbreviated paper title here
%
\author{First Author\inst{1}\orcidID{0000-1111-2222-3333} \and
Second Author\inst{2,3}\orcidID{1111-2222-3333-4444} \and
Third Author\inst{3}\orcidID{2222--3333-4444-5555}}
%
\authorrunning{F. Author et al.}
% First names are abbreviated in the running head.
% If there are more than two authors, 'et al.' is used.
%
\institute{affiliation 1 \and
Springer Heidelberg, Tiergartenstr. 17, 69121 Heidelberg, Germany
\email{lncs@springer.com}\\
\url{http://www.springer.com/gp/computer-science/lncs} \and
ABC Institute, Rupert-Karls-University Heidelberg, Heidelberg, Germany\\
\email{\{abc,lncs\}@uni-heidelberg.de}}
%
\maketitle              % typeset the header of the contribution
%
\begin{abstract}
last to be written
\keywords{Twitter analytics  \and online user discovery \and online activists \and influencers}
\end{abstract}
%


%
%
\section{Introduction}


\subsection{Motivation: discovering online activists}
\anote{The intro starts with motivation (online activism) }

\anote{serves as general motivation --- see above. Who are ``online activists'' and why is it interesting to identify them?}
According to the Cambridge Dictionary, an \textit{activist} is  ``A person who believes strongly in political or social change and takes part in activities such as public protests to try to make this happen''.
In contrast, the Dictionary defines an influencer as ``Someone who affects or changes the way that other people behave, for example through their use of social media''.
%
While a large of body of research has been generated over the past ten years on techniques for finding influencers online \cite{RIQUELME2016949}, the problem of identifying activists has received less attention. 
While online activism is a well-known phenomenon \cite{IJoC1246}, research has been limited to the study of its broad societal impact. 
In contrast, we are interest in the fine-grained discovery of activists at the level of the single individual, in other words, we seek to identify people who feel passionate about a cause or topic, and who take action for it.
\anote{activism in social movement literature:  \cite{doi:10.1080/14742830701497277}} 

\anote{a critique of established network methods to identify these particular profiles:
\begin{itemize}
	\item activists are users who are interesting but not in the same sense as influencers (many defs). but these are ``low-key'' users who are not famous or at the centre of large networks. 
	\item Thus discovering such users is hard because they ``disappear in the noise'' of a large global social network with a very broad range of topics.
	\item traditional approaches don't work well for these profiles. (this is a bit axiomatic - not proven)
\end{itemize}
}

\anote{why influencers are easy and activists are hard -- problem is signal/noise ratio}

Our operational definition of activists is related to, but different from, that of \textit{influencers} in a social network. 
For example, influencers are typically people who are well-known either online or in the real world, and who acts as hubs and amplifiers for messages associated with specific topics. 
In contrast, activists are, by our own definition, not famous people, rather they are distinguished by their active attitude towards social engagement, either with a special focus on specific issues, or broadly across a range of issues.

\anote{Activists presence in social media is widely acknowledged. Social media facilitates activists communication and organization \cite{Poell2014}.  Some examples on how social media enable activism are:
	Cause and topic awareness;	Social gathering and activities organization;
	Organize help efforts and diffusion of information during disasters.\\
	
	Social media empowers activists by \cite{Youmans2012}:
	Easing public acting in coordination for disaffected citizens; 	Creating information cascades that bolstered protesters' perceptions of the likelihood of success;

}


In contrast, in our current incremental approach, the idea is to first discover one or more online contexts asssociated with a topic, such as a social campaign or a local event.
Our hypothesis is that each such  context provides a low-noise scope within which those users who would not otherwise be identified using standard network analysis techniques, are given a chance to ``stand out'' and emerge, as the limited scope may amplify their otherwise weak signal and isolate them from some of the global participants.
We also make the hypothesis that, within each such focused context, we maybe able to  successfully use established social network metrics to identify users who are active within the context, i.e., those who engage in meaningful interactions that provide a limited but strong signal about their interest in the chosen topic. 
%
As context are temporally (and optionally also spatially) characterised, this approach also lends itself well to incremental user discovery,  and to building strong profiles of users who are observed in multiple contexts over time.


\anote{Zika angle probably too specific so the below is parked:
\begin{itemize}

\item Mosquito-borne diseases such as Zika, Dengue and Chikungunya are endemic in several countries and Brazil is one of the most affected ones.
\item These diseases cause severe symptomatic effects and complications which include hemorrhagic fever and microcephaly in newborns of contaminated mothers.
\item Why they are difficult to fight: No vaccines available, Government is ineffective, slow and has low resources
\end{itemize}

Our long-term goal is to help alleviate these problems by supporting the health officials in their efforts to detect and clean up mosquito breeding sites.
Thus, in a practical sense our study is on techniques for identifying those individuals who are likely to become engaged with volunteering efforts.
Note that we are not seeking people who have shown  specific interest in the fight against Zika or other epidemics, but rather, more generally, we are looking for people who have a recent and sustained history of social engagement. 
%
}


\subsection{Reference case study and running example}  \label{sec:reference}

\anote{where we are looking for activists. This case study is intended to meet our original motivation, while demonstrating the user discovery process in action. There is no separate ground truth validation so this intends to show how the framework works when you choose topics for the contexts, and a specific set of metrics.}

\subsection{Overview of the approach }

\anote{this may not be needed?  depending on space}

\anote{
	the definition of activists is based on a few established metrics. we do not invent new metrics but we combine a few that exist in a way that is justified intuitively, by drawing from literature on online activism.

the metrics include (1) content-based, non-topological metrics based on a user's post history

(2) non-contextual topological

(3) non-contextual topological.

in past approaches, specific profiles, such as those of influencers and many others \cite{RIQUELME2016949}, have been proposed. these essentially combine a set of basic metrics.

call these \textit{engineered features}. 
}


\anote{Candidate discovery is based on a weak notion of online event. It is \textit{weak} because on Twitter events are not explicitly defined, unlike for example in Facebook or Meetup.}

\anote{Our technique achieves two things:

\begin{itemize}
\item discover candidate Twitter profiles
\item repeatedly measure the \textit{strength} of candidate profiles as activists over time, thus enabling dynamic profile ranking
\end{itemize}	
}

\anote{clarify that when we say ``social media'' in this paper we refer exclusively to Twitter. So I guess just as well say Twitter explicitly everywhere? }


\subsection{Paper contributions}

The paper offers the following specific contributions.
\begin{itemize}
\item We propose a semi-automated, incremental process for identifying \textit{low-profile} users who are characterised by a weak online signal, i.e., they are not ``global influencers'', and yet they are interesting as they show interest in actively engaging with social issues online. 

\item As a concrete example of the process in use, we characterise our target users as ``online activists '' by proposing an operational definition of their profiles in terms of well-established, quantifiable Twitter metrics which can be used;

\item A reference implementation of the process and its demonstration on a reference case for online activism in the \hl{XX} domain;

\item A performance evaluation to assess the impact of the Twitter API limitations on content retrieval, specifically sampling from the ``garden hose'' 

\end{itemize}

\subsection{Related Work}

\anote{
	\begin{itemize}
		\item 	generically on finding influencers online. 
		\item \hl{WHAT ELSE??}
			\end{itemize}
	
  }


%%%%%%%%%%%%%%%%%%%%%
\section{Technical Approach}
%%%%%%%%%%%%%%%%%%%%%


As mentioned in the introduction, the main limitation of our own previous approach 	\cite{Missier2017,Sousa2018,Barros2018} is that topology-based metrics such as h-index and centrality analysis could not be used, because the users identified through the content relevance approach led to collections of disconnected users. Essentially, each user would be considered relevant in isolation, but on the basis of a very sparse signal.
%
A broader motivation for the approach we advocate here is that relatively low-profile users are hard to identify by using traditional approaches based on global content harvesting and large Twitter networks.



\anote{
	The main insight behind the process structure is to first identify suitable contexts (online campaigns, events around social issues) where the target user profiles stand out relative to the global network. 
	
	Context discovery is currently the only step where human intervention is required (but that can be crowd-sourced as it does not involve deep specialist knowledge).
	With some constraints that are clearly identified below, a variety of Twitter-based metrics for characterising users 
}

\comment{Since our method relies on small networks that represent limited topical context, our challenge is to find metrics that operate well on small networks.}{rephrase}

	\anote{
		\begin{itemize}
			\item 	 	 Weak events provide the necessary context. 
			\item Observing users in context has the effect of drastically reducing the noise of general Twitter signal, relative to the general stream.  
			\item Conversations that take place within the context of an event are represented as dynamic weighted social network graphs. 
			\item Depending on the size and importance of the event, these networks may be first partitioned into smaller virtual overlapping communities, using the Daemon community detection algorithm
			\item users within each community (for small communities, we use the entire network) are then ranked according to their topolgical properties
			\item for the top-\textit{k} users within each community, we compute activism metrics and derive a strength of their profile.
			\item Progressively build up stronger profiles of repeat users, by updating their strength using the new values for the metrics
		\end{itemize}
	}
	
	
\subsection{Contexts}

As described In Sec. ~\ref{sec:reference}, contexts are meant to identify events or campaigns around social issues, which are characterised by spatio- and/or temporal boundaries and by hashtags and/or keyword terms.
In the Twitter settings we refer to these as \textit{weak} contexts, because Twitter does not natively support the notion of event or campaign (unlike, for example, Facebook, Instagram, or Meetup).
In abstract, we represent as generic context as
\begin{equation}
    C = \langle s, [t_1, t_2], K \rangle 
    \label{eq:context}
\end{equation}
where $s$ represents a bounding box, $[t_1, t_2]$ a time intervals and $K = \{ k_1 \dots k_n\}$ is the set of terms used to filter content within the spatio-temporal boundaries.

$C$ defines search criteria, which produce a set $T$ of tweets when submitted to Twitter.
Each tweet $t \in T$ is an instance of one of these possible activities: an \textit{original tweet}, a \textit{retweet}, and a \textit{mention}.
Let $u(t)$ be the user who originated $t$.
We say that both $t$ and  $u(t)$ are within context $C$.

We also define the complement $\Tilde{T}$ of $T$ as the set of tweets produced by omitting the terms $K$ from $C$, that is, those retrieved solely using the spatio-temporal criteria. 
We refer to these tweets, and their respective users, as out-of-context $C$.

\subsection{Online activists: a broad operational definition}

\anote{this is too specific on the case study. We must say that the framework accommodates a variety of metrics, provided that they fall into the  three categories that we have identified. Use the specific metrics as examples}

The impressive amount of  Twitter user roles that have been proposed in the literature are all based on a few fundamental metrics and a few algorithms that operate on the Twitter network \cite{RIQUELME2016949}.
To the best of our knowledge, the notion of \textit{online activist} has only been used in the context of the broad societal impact of online activism, but not to characterise  individual users.
Thus, we begin by proposing our own operational definition of the \textit{online activist} role, with the understanding that what matters in this work is not so much the specific choice of metrics and their combination, but rather how they are computed within limited contexts, such as specific events or campaigns as in our case study.
Specifically, we distinguish between three types of metrics, namely:
\begin{enumerate}
    \item those that rely solely on content and not on the user-user graph topology. These metrics are defined relative to a topic of interest, which in our framework is a context;
\item topological metrics that encode context-independent, long-lived relationships amongst users; and 
\item topological metrics that encode user relationships that occur within a context.
\end{enumerate}

Our choice of exemplars for each of these classes follows the indications found in recent studies on online activism, namely \cite{Lotan2011} and  \cite{Poell2014}.
The specific chosen metrics are as follows.

\paragraph{Content-based metrics} are \textit{Retweeting Rate}, \textit{Retweeted Rate}, and  \textit{Topical Attachment}, proposed in \cite{Bizid:2015} (the use of \textit{Topical Attachment} was suggested in \cite{Poell2014}).

For a user $u$ within context $C$, the \textit{Retweeting Rate}  $CM_1$ is intended to measure the impact of  original tweets posted  by other users on $u$'s activity within the context, while the \textit{Retweeted Rate} $CM_2$  measures the impact of original tweets produced by $u$ within the context, on other users.
Formally, given a context $C$ containing user $u$, we define:

\noindent 
$R1(u)$: The number of retweets by $u$, of tweets from other in-context users;\\
$R2(u)$: The number of unique users in $C$, who have been retweeted by $u$; \\
$R3(u)$: The number of retweets of $u$'s  tweets; \\
$R4(u)$: The number of unique users in $C$ who retweeted $u$'s tweets; \\
$T1(u)$: The number of original tweets posted by $u$ within $C$; \\
$T1(u)$: The total number of web links found in original tweets posted by $u$ within $C$.

Furthermore, each of these base metrics  are qualified to indicate whether or not the activity is within or out of context.
For instance, we write $R1_{on}(u)$ to denote the number of context retweets and $R1_{off}(u)$ the number of out-of-context retweets by $u$, i.e., these are retweets that occur within $C$'s spatio-temporal boundaries, but do not contain any of the hashtags or keywords that define $C$.  
Using this notation, the \textit{Retweeting Rate} $RR_1(u)$, \textit{Retweeted Rate} $RR_2(u)$ and \textit{Topical Attachment} $TA(u)$ for each $u$ are defined as follows~\cite{Bizid:2015}:
\begin{align}
RR_1(u) & =  R1_{on}(u) \cdot \log(R2_{on}+1)(u) - R1_{off}(u) \cdot \log(R2_{off}(u)+1) \\
RR_2(u) & =  R3_{on}(u) \cdot \log(R3_{on}+1)(u) - R3_{off}(u) \cdot \log(R4_{off}(u)+1) \\
TA(u) & = \frac{T1_{on}(u) + T2_{on}(u)}{T1_{off}(u) + T2_{off}(u) +1} 
\end{align}

We only use one context-independent topological metric, namely the  FollowerRank $FR(u)$, defined for instance in \cite{RIQUELME2016949} as:
\begin{equation}
FR(u) = \frac{F1(U)}{F2(u)}
\end{equation}
where $F1(u)$ and $F2(u)$ are the number of followers and followees of $u$.

Finally, we use \textit{in-degree centrality} (one of the many measures of centrality listed in \cite{RIQUELME2016949}) as our only context-specific topological metric, defined as:
\begin{equation}
    IC(u) = \frac{\mathit{indegree}(u)}{N-1}
\end{equation}
for each user $u$, where $N$ is the number of nodes in the network induced by $C$.



	
\subsection{Users discovery process framework}   \label{sec:discovery}

\anote{ it is a framework because there are important parameters: 
	\begin{itemize}
\item choice of specific metrics
\item choice of function to combine the metrics for user ranking. 
\item ranking thresholds
\item white and black lists of users to remove from the top-k results
\item context discovery criteria (which are domain-specific)
	\end{itemize}

}


\subsection{Prototype architecture}	 

	\anote{see ``Pipeline architecture'' in the presentation}
	
	
\section{Experimental evaluation}

\anote{here we note that functional evaluation was partially carried out in the previous section, while illustrating the pipeline. 
	What we add here: 

\begin{itemize}
  \item comparing users found with influencers on the same case study dataset
  \item comparing different approaches to community detection and user ranking
\end{itemize}

performance: discuss limitations of using the garden hose for content harvesting  (here or in next section??)\\


do we need more than one case study?  My take is that one is good enough for a conference paper --- to be extended later

}
	
	
\section{Discussion and ongoing work}

\anote{
	we say that events are manually identified. Sketch the events bootstrapping idea.
}

 \bibliographystyle{splncs04}
 \bibliography{icwe19}

\end{document}
